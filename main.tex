\documentclass[a4paper,11pt]{report}
\usepackage[T1]{fontenc}
\usepackage[utf8]{inputenc}
\usepackage{lmodern}
\usepackage[french]{babel}
\usepackage{fullpage}
\usepackage{tikz-uml}
\usetikzlibrary{positioning}

\title{GEN - Génie logiciel\\Notes de cours}
\author{Gabriel Roch}
\date{Semestre de printemps 2020}

\begin{document}

\maketitle
\tableofcontents


\begin{abstract}
\end{abstract}

\chapter{Cycle de vie}
\paragraph{Spécification} Défini ce que le logiciel doit faire (les besoins et \emph{pas} les solutions).
\paragraph{Conception} Défini l’organisation du système (architecture du système et classes OO).
\paragraph{Mise en oeuvre} Écriture du code.
\paragraph{Validation} Vérifie que le logiciel répond aux spécifiation et aux attentes du client
\paragraph{Évolution} Modifie le système en réponse aux nouveaux besoins

\section{Modèle en cascade}

\begin{figure}[h]
  \begin{center}
    \begin{tikzpicture}[scale=1]
      \umlusecase[x=0.0,y=-0,width=2.5cm,name=requir]{Requirements definition}
      \umlusecase[x=2.5,y=-2,width=2.5cm,name=design]{System and software design}
      \umlusecase[x=5.0,y=-4,width=2.5cm,name=implem]{Implementation and unit testing}
      \umlusecase[x=7.5,y=-6,width=2.5cm,name=integr]{Integration and system testing}
      \umlusecase[x=10,y=-7.5,width=2.5cm,name=operat]{Operation and maintenance}

      \umluniassoc[geometry=-|]{requir}{design}
      \umluniassoc[geometry=-|]{design}{implem}
      \umluniassoc[geometry=-|]{implem}{integr}
      \umluniassoc[geometry=-|]{integr}{operat}
      \umluniassoc[geometry=-|]{operat}{integr}
      \umluniassoc[geometry=-|]{operat}{implem}
      \umluniassoc[geometry=-|]{operat}{design}
      \umluniassoc[geometry=-|]{operat}{requir}
      
      \umlnote[y=+0,x=5.5,width=4.25cm]{requir}{Document de spécification, le plus détaillé possible}
      \umlnote[y=-2,x=8.75,width=5.5cm]{design}{Architecture du système et conception détaillée: modules, API, ...}
      \umlnote[y=-4,x=10,width=3cm]{implem}{Code du logiciel}
      \umlnote[y=-6,x=13,width=2.4cm]{integr}{Version livrable du logiciel}
    \end{tikzpicture}
    \caption{Modèle en cascade}
    \label{fig:waterfall-001}
  \end{center}
\end{figure}

\paragraph{Avantage}
\begin{itemize}
  \item Développement linéaire
  \item Permet la plannification et la coordination de grandes équipes à distance
\end{itemize}
\paragraph{Incovéniant}
\begin{itemize}
  \item Mal adapter au changement en cours de projet
  \item Les phases se font l'une après l'autre
  \item Le clients ne voit le résultat que tout à la fin
\end{itemize}

\section{Cycle en V}


\end{document}
